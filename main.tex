% !TEX program = xelatex

\documentclass[10pt,a4paper]{article}
\usepackage[top = 1.5cm, bottom = 1.5cm, left = 1.5cm, right = 1.5cm]{geometry}

\usepackage{titling}
\usepackage[czech]{babel}
\usepackage{graphicx}
\usepackage{lmodern}
\usepackage{hyperref}
\usepackage{setspace}
\usepackage{csvsimple}

\usepackage{amsmath}
\usepackage{amssymb}
\usepackage{gensymb}
\usepackage{units}
\usepackage{bm}
\delimitershortfall=-1pt

\usepackage{mathtools}
\usepackage{accents}
\usepackage{calc}

% no page break
\newenvironment{absolutelynopagebreak}
  {\par\nobreak\vfil\penalty0\vfilneg
   \vtop\bgroup}
  {\par\xdef\tpd{\the\prevdepth}\egroup
   \prevdepth=\tpd}


% redefine \sqrt
\usepackage{letltxmacro}
\makeatletter
\let\oldr@@t\r@@t
\def\r@@t#1#2{%
\setbox0=\hbox{$\oldr@@t#1{#2\,}$}\dimen0=\ht0
\advance\dimen0-0.2\ht0
\setbox2=\hbox{\vrule height\ht0 depth -\dimen0}%
{\box0\lower0.4pt\box2}}
\LetLtxMacro{\oldsqrt}{\sqrt}
\renewcommand*{\sqrt}[2][\ ]{\oldsqrt[#1]{#2\,}\,}
\makeatother

% redefine \hbar
\LetLtxMacro{\oldhbar}{\hbar}
\renewcommand*{\hbar}{{\mathpalette\hbaraux\relax\mathrm{h}}}
\newcommand*{\hbaraux}[2]{\sbox0{\mathsurround=0pt$#1\mathchar'26$}\mkern-1mu\lower.07\ht0\box0\mkern-8mu}

\def\ph{\phantom}
\def\vph{\vphantom}
\def\hph{\hphantom}
\def\rzw{\mathrlap}
\def\lzw{\mathllap}
\def\czw{\mathclap}

\newcommand{\nph}[1]{\settowidth{\dimen0}{#1}\hspace*{-\dimen0}}

\newcommand*{\mask}[2]{%
    \mathord{\makebox[\widthof{\(#1\)}]{\(#2\)}}%
}

\def\?{\mathit{?}}

\newcommand{\comm}[2]{\left[ #1, #2 \right]}
\newcommand{\const}[1]{\text{#1}}
\newcommand{\norm}[1]{\left\lVert#1\right\rVert}

\newcommand{\mat}[1]{
    \begin{pmatrix}
        #1
    \end{pmatrix}
}

\newcommand{\mata}[2]{
    \left(
    \begin{array}{@{}#1@{}}
        #2
    \end{array}
    \right)
}

\newcommand{\smat}[2][1]{
    \scalebox{#1}{$\mat{#2}$}
}

\newcommand{\abs}[1]{\left| #1 \right|}

\renewcommand{\d}[1]{\;\const{d}#1}
\newcommand{\dd}[2]{\frac{\const{d} #1}{\const{d} #2} \;}
\newcommand{\pd}[2]{\frac{\partial  #1}{\partial  #2} \;}

\newcommand{\bra}[1]{\left< #1 \right|}
\newcommand{\ket}[1]{\left| #1 \right>}
\newcommand{\braket}[2]{\left< #1 \middle| #2 \right>}

\newcommand{\e}[1]{\const{e}^{#1}}
\renewcommand{\i}{\const{i}}

\def\R{\mathbb{R}}
\def\C{\mathbb{C}}

\def\1{\const{Id}}

\newcommand{\bigdot}[1]{\accentset{\bullet}{#1}}


\def\kzero{\ket{\mask{+}{0}}}
\def\kone{\ket{\mask{+}{1}}}
\def\ktwo{\ket{\mask{+}{2}}}
\def\kplus{\ket{+}}
\def\kminus{\ket{-}}

\def\bone{\bra{\mask{+}{1}}}
\def\btwo{\bra{\mask{+}{2}}}

\begin{document}

\title{Kvantová mechanika}
\author{Michal Grňo}
\date{\today}

\maketitle

\underline{Definice:}
Mějme stav $\ket{\psi(t)}$, který se vyvíjí v čase $t$ a označme $\ket{\psi} \equiv \ket{\psi(0)}$. Potom existuje operátor $\hat{U}(t)$, pro který platí $$\ket{\psi(t)} = \hat{U}(t) \ket{\psi} \: .$$ Operátor $\hat{U}(t)$ je lineární (PROČ?) a nazýváme ho \textbf{evoluční operátor stavu} $\ket{\psi}$. Triviálně platí $\hat{U}(0) = \1$.

\vspace{2em}
\underline{Definice:}
Nechť pro každé dva stavy $\ket{\psi}, \ket{\phi}$ platí $$\abs{\braket{\psi}{\phi}}^2 = \abs{\braket{\psi(t)}{\phi(t)}}^2 \quad \forall t,$$ potom říkáme, že \textbf{dynamika systému je nezávislá na čase}.

\vspace{2em}
\underline{Důsledek:}
Je-li dynamika systému nezávislá na čase, potom je evoluční operátor každého stavu \textbf{unitární} a existuje samoadjugovaný operátor $\hat{H}$, že $$\hat{U}(t) = \e{-\i \, \hat{H}(t) \, t}.$$

\underline{Důkaz:}
Z nezávislosti dynamiky na čase plyne, že pro každý stav $\ket{\psi}$ platí $$\abs{\braket{\psi}{\psi}}^2 = \abs{\braket{\psi(t)}{\psi(t)}}^2 = \abs{\bra{\psi} \, \hat{U}^+\!(t) \; \hat{U}(t) \, \ket{\psi}}^2,$$ z rovnosti absolutních hodnot vyplývá, že se vnitřky rovnají až na nějaké jednotkové komplexní číslo, tedy existuje takové $s(t) \in \R$, že $$\braket{\psi}{\psi} = \bra{\psi} \, \hat{U}^+\!(t) \; \hat{U}(t) \, \ket{\psi} \; \e{\i \, s(t)}.$$ Vidíme ale, že obecně $\e{\i \, s(t)}$ nejde rozdělit mezi $\hat{U}$ a $\hat{U}^+$: každé komplexní číslo, kterým vynásobíme $\hat{U}$ se nám v $\hat{U}^+$ vrátí jako komplexně sdružené. Musí tedy platit $\e{\i \, s(t)} = \pm 1$. Protože navíc požadujeme, aby rovnost platila v čase $t=0$ a vývoj systému byl spojitý, zbývá nám pouze $+1$. Pro evoluční operátor z toho plyne $$\braket{\psi}{\psi} = \bra{\psi} \, \hat{U}^+\!(t) \; \hat{U}(t) \, \ket{\psi},$$ $$\hat{U}^+ \!(t) \; \hat{U}(t) = \1,$$ tedy pro každé $t$ je evoluční operátor unitární.

Z lineární algebry víme, že každý unitární operátor lze vyjádřit jako komplexní exponenciolu samoadjugovaného operátoru (I~PRO NEKONEČNOU DIM?), tedy existuje takový samoadjugovaný operátor $\hat{A}(t)$, že $$\hat{U}(t) = \e{-\i \, \hat{A}(t)}.$$ Navíc máme požadavek, aby $\hat{U}(0) = \1$. Tento požadavek bychom mohli naplnit tak, že budeme požadovat, aby $\hat{A}(t) = 0$, ukazuje se však, že přirozenější je zavést samoadjugovaný operátor $\hat{H}(t)$, že $\hat{A}(t) = t \, \hat{H}(t)$, tedy $$\hat{U}(t) = \e{-\i \, \hat{H}(t) \, t}.$$

\vspace{2em}
\underline{Poznámka:}
Z matematického hlediska jsme místo $\e{-\i \, \hat{H}(t) \, t}$ mohli stejně dobře psát $\e{+\i \, \hat{H}(t) \, t}$. Proč jsme zvolili mínus bude zřejmé, až objevíme fyzikální význam operátoru $\hat{H}$.

\vspace{2em}
\underline{Důsledek:}
Nechť je dynamika systému nezávislá na čase, potom platí tzv. \textit{časová Schrödingerova rovnice} (TDSE):
$$ \i \; \dd{}{t} \! \ket{\psi(t)} = \hat{H}(t) \ket{\psi(t)} $$

\underline{Důkaz:}
Víme, že platí $$\ket{\psi(t)} = \hat{U}(t) \ket{\psi} = \e{-\i \, \hat{H}(t) \, t} \ket{\psi},$$ nyní rovnici zderivujeme podle času $$\dd{}{t} \ket{\psi(t)} = -\i \, \hat{H}(t) \e{-\i \, \hat{H}(t) \, t} \ket{\psi} = -\i \, \hat{H}(t) \ket{\psi(t)}.$$

\vspace{2em}
\underline{Definice:}
Mějme stav $\ket{\psi}$, pro který platí $$\abs{\braket{\psi}{\psi(t)}}^2 = 1 \quad \forall t.$$ Takový stav nazýváme \textbf{stacionárním stavem systému}.

\vspace{2em}
\underline{Důsledek:}
Pro každý stacionární stav existuje $E(t) \in \R$, že $$\ket{\psi(t)} = \e{-\i \, E(t)} \ket{\psi},$$ totiž $$\abs{\braket{\psi}{\psi}}^2 = 1 = \abs{\braket{\psi}{\psi(t)}}^2 = \abs{\bra{\psi} \hat{U}(t) \ket{\psi}}^2,$$ $$\braket{\psi}{\psi} = \bra{\psi} \hat{U}(t) \ket{\psi} \e{-\i E(t)},$$ $$\hat{U}(t) = \e{-\i E(t)}.$$


\end{document}
