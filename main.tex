% !TEX program = xelatex

\documentclass[10pt,a4paper]{article}
\usepackage[top = 1.5cm, bottom = 1.5cm, left = 1.5cm, right = 1.5cm]{geometry}

\usepackage{titling}
\usepackage[czech]{babel}
\usepackage{graphicx}
\usepackage{lmodern}
\usepackage{hyperref}
\usepackage{setspace}
\usepackage{csvsimple}

\usepackage{amsmath}
\usepackage{amssymb}
\usepackage{gensymb}
\usepackage{amsthm}
\usepackage{units}
\usepackage{bm}
\delimitershortfall=-1pt

\usepackage{mathtools}
\usepackage{accents}
\usepackage{calc}

% aby se \subsection nečíslovalo
\setcounter{secnumdepth}{1}

% no page break
\newenvironment{absolutelynopagebreak}
  {\par\nobreak\vfil\penalty0\vfilneg
   \vtop\bgroup}
  {\par\xdef\tpd{\the\prevdepth}\egroup
   \prevdepth=\tpd}


% redefine \sqrt
\usepackage{letltxmacro}
\makeatletter
\let\oldr@@t\r@@t
\def\r@@t#1#2{%
\setbox0=\hbox{$\oldr@@t#1{#2\,}$}\dimen0=\ht0
\advance\dimen0-0.2\ht0
\setbox2=\hbox{\vrule height\ht0 depth -\dimen0}%
{\box0\lower0.4pt\box2}}
\LetLtxMacro{\oldsqrt}{\sqrt}
\renewcommand*{\sqrt}[2][\ ]{\oldsqrt[#1]{#2\,}\,}
\makeatother

% redefine \hbar
\LetLtxMacro{\oldhbar}{\hbar}
\renewcommand*{\hbar}{{\mathpalette\hbaraux\relax\mathrm{h}}}
\newcommand*{\hbaraux}[2]{\sbox0{\mathsurround=0pt$#1\mathchar'26$}\mkern-1mu\lower.07\ht0\box0\mkern-8mu}


\newtheorem{theorem}{Věta}[section]
\theoremstyle{definition}
\newtheorem{definition}[theorem]{Definice}
\newtheorem{lemma}[theorem]{Lemma}
\newtheorem*{remark}{Poznámka}
\newtheorem*{example}{Příklad}
\newtheorem{corollary}[theorem]{Důsledek}
\newtheorem*{exercise}{Cvičení}

\def\ph{\phantom}
\def\vph{\vphantom}
\def\hph{\hphantom}
\def\rzw{\mathrlap}
\def\lzw{\mathllap}
\def\czw{\mathclap}

\newcommand{\nph}[1]{\settowidth{\dimen0}{#1}\hspace*{-\dimen0}}

\newcommand*{\mask}[2]{%
    \mathord{\makebox[\widthof{\(#1\)}]{\(#2\)}}%
}

\def\?{\mathit{?}}

\newcommand{\comm}[2]{\left[ #1, #2 \right]}
\newcommand{\const}[1]{\mathrm{#1}}
\newcommand{\norm}[1]{\left\lVert#1\right\rVert}

\newcommand{\mat}[1]{
    \begin{pmatrix}
        #1
    \end{pmatrix}
}

\newcommand{\mata}[2]{
    \left(
    \begin{array}{@{}#1@{}}
        #2
    \end{array}
    \right)
}

\newcommand{\smat}[2][1]{
    \scalebox{#1}{$\mat{#2}$}
}

\newcommand{\abs}[1]{\left| #1 \right|}

\renewcommand{\d}[1]{\;\const{d}#1}
\newcommand{\dd}[2]{\frac{\const{d} #1}{\const{d} #2} \;}
\newcommand{\pd}[2]{\frac{\partial  #1}{\partial  #2} \;}

\newcommand{\bra}[1]{\left< #1 \right|}
\newcommand{\ket}[1]{\left| #1 \right>}
\newcommand{\braket}[2]{\left< #1 \middle| #2 \right>}

\newcommand{\innerprod}[2]{\big( #1, #2 \big)}
\newcommand{\duality}[2]{\big< #1, #2 \big>}

\newcommand{\e}[1]{\const{e}^{#1}}
\renewcommand{\i}{\const{i}}

\def\R{\mathbb{R}}
\def\C{\mathbb{C}}
\def\H{\mathcal{H}}

\def\1{\hat{I}}

\newcommand{\bigdot}[1]{\accentset{\bullet}{#1}}


\def\kzero{\ket{\mask{+}{0}}}
\def\kone{\ket{\mask{+}{1}}}
\def\ktwo{\ket{\mask{+}{2}}}
\def\kplus{\ket{+}}
\def\kminus{\ket{-}}

\def\bone{\bra{\mask{+}{1}}}
\def\btwo{\bra{\mask{+}{2}}}

\def\konst{\mathrm{konst.}}

\begin{document}

\title{Kvantová mechanika bez mávání rukama}
\author{Michal Grňo}
\date{\today}

\maketitle

\section{Hrátky s operátory}
\begin{lemma}[Polarizační identita]
    \label{polarizacni-identita}
    Nehcť $\H$ je komplexní Hilbertův prostor, potom pro všechny $\psi, \phi \in \H$ platí
    \begin{equation*}
        \innerprod{\psi}{\phi}_\H
        =
        \frac{1}{4}
        \left(
            \norm{\psi + \phi}^2
            - \norm{\psi - \phi}^2
            + \i \norm{\psi + \i \phi}^2
            - \i \norm{\psi - \i \phi}^2
        \right)
    \end{equation*}
\end{lemma}
\begin{proof}
    Rozepsáním pomocí vztahu $\norm{\phi}^2 = \innerprod{\phi}{\phi}$ lze snadno ukázat, že
    \begin{equation*}
        \norm{\psi + \phi}^2 - \norm{\psi - \phi}^2
        = 4 \Re \, \innerprod{\psi}{\phi},
        \quad
        \norm{\psi + \i\phi}^2 - \norm{\psi - \i\phi}^2
        = 4 \Im \, \innerprod{\psi}{\phi}.
    \end{equation*}
\end{proof}

\begin{lemma}[O unitaritě operátoru zachovávajícího normu]
    \label{unitarni-zachova-normu}
    Mějme komplexní Hilbertův prostor $\H$ a na něm lineární operátor $\hat T$ definovaný na $\mathrm{D}(\hat T) \subset \H$, potom platí
    \begin{equation*}
        \norm{\hat T \phi} = \norm{\phi}
        \;\forall \phi \in \mathrm{D}(\hat T)
        \iff
        \hat T \text{ je unitární.}
    \end{equation*}
\end{lemma}
\begin{proof}
    % https://math.stackexchange.com/q/1626701
    Implikaci „$\Leftarrow$“ ukážeme snadno:
    \begin{equation*}
        \norm{\hat T \phi}^2
        = \innerprod{\hat T \phi}{\hat T \phi}
        = \innerprod{\hat T^+ \hat T \phi}{\phi}
        = \innerprod{\phi}{\phi}
        = \norm{\phi}^2
    \end{equation*}
    Implikace opačným směrem potom plyne z \ref{polarizacni-identita}. Buďte $\psi, \phi \in \H$, ukážeme, že $T$ zachovává skalární součin:
    \begin{align*}
        4 \, \innerprod{\hat T \psi}{\hat T \phi}
        &= \norm{\hat T \psi + \hat T \phi}^2
        - \norm{\hat T \psi - \hat T \phi}^2
        + \i \norm{\hat T \psi - \i \hat T \phi}^2
        - \i \norm{\hat T \psi + \i \hat T \phi}^2
        \\[10pt]
        &= \norm{\hat T(\psi + \phi)}^2
        - \norm{\hat T(\psi - \phi)}^2
        + \i \norm{\hat T(\psi - \i \phi)}^2
        - \i \norm{\hat T(\psi + \i \phi)}^2
        \\[10pt]
        &= \norm{\psi + \phi}^2
        - \norm{\psi - \phi}^2
        + \i \norm{\psi - \i \phi}^2
        - \i \norm{\psi + \i \phi}^2
        = 4 \, \innerprod{\psi}{\phi}
    \end{align*}
\end{proof}


\section{Vektorový formalismus}

\subsection{Fyzikální motivace}
TODO.

\subsection{Matematický framework}
Je zvykem popisovat kvantové systémy pomocí abstraktních vektorů z nějakého Hilbertova prostoru, jehož dimenze závisí na modelovaném problému. V praxi pracujeme s všemožnými prostory od $\C^2$ pro popis nejjednoduších dvouhladinových systémů přes $W^{1,2}(\R)$ pro popis volné částice až po neseparabilní prostory. Kvantové stavy jsou představovány jednotkovými vektory z těchto prostorů.

\begin{definition}[Evoluční operátor]
    Mějme stav $\ket{\psi(t)}$, který se vyvíjí v čase $t$ a označme $\ket{\psi} \equiv \ket{\psi(0)}$. Potom existuje operátor $\hat{U}(t)$, pro který platí $$\ket{\psi(t)} = \hat{U}(t) \ket{\psi} \: .$$ Operátor $\hat{U}(t)$ je lineární (PROČ?) a nazýváme ho evolučním operátorem. Triviálně platí $\hat{U}(0) = \1$.
\end{definition}

\begin{definition}[Dynamika nezávislá na čase]
    Nechť pro každé dva stavy $\ket{\psi}, \ket{\phi}$ platí $$\abs{\braket{\psi}{\phi}}^2 = \abs{\braket{\psi(t)}{\phi(t)}}^2 \quad \forall t,$$ potom říkáme, že dynamika systému je nezávislá na čase.
\end{definition}

\begin{lemma} \label{dyn-sys-charakteristika}
    Tyto tři výroky jsou ekvivalentní:
    \begin{enumerate}
        \item Dynamika systému je nezávislá na čase
        \item Evoluční operátor $\hat U$ je v každém čase unitární
        \item Existuje samoadjugovaný operátor $\hat A(t)$ takový, že $\hat{U}(t) = \e{-\i \, \hat{A}(t) \, t}$.
    \end{enumerate}
\end{lemma}
\begin{proof}
    Rozepíšeme si definici prvního výroku:
    \begin{equation*}
        \abs{\braket{\psi}{\phi}}^2 = \abs{\braket{\psi(t)}{\phi(t)}}^2 = \abs{\bra{\psi} \, \hat{U}^+\!(t) \; \hat{U}(t) \, \ket{\phi}}^2.
    \end{equation*}
    Implikace $1 \Leftarrow 2$ je hned zřejmá, nyní dokážeme implikaci $1 \Rightarrow 2$.

    Protože rovnost platí pro každé $\ket{\psi}, \ket{\phi}$, můžeme volit i $\ket{\psi} = \ket{\phi}$:
    \begin{equation*}
        \norm{\ket\psi}^2
        \equiv \braket{\psi}{\psi}
        = \braket{\psi(t)}{\psi(t)}
        \equiv \norm{\ket{\psi(t)}}^2
        = \norm{\hat U \ket\psi}^2
    \end{equation*}
    Z \ref{unitarni-zachova-normu} potom víme, že operátor, který zachovává normu, je nutně unitární.

    Nakonec dokážeme ekvivalenci $2 \Leftrightarrow 3$. Z lineární algebry víme, že pro každý unitární operátor $\hat U$ existuje samoadjugovaný operátor $\hat B$ takový, že $\hat U = \e{ \i \hat B}$, a naopak, že pro každý samoadjugovaný operátor $\hat B$ je výraz $\e{ \i \hat B}$ unitární. Pro evoluční operátor tedy máme
    \begin{equation*}
        \hat U(t) = \e{\i \, \hat B(t)}.
    \end{equation*}
    Protože navíc po evolučním operátoru požadujeme, aby $\hat U(0) = \1$, musí platit $\hat B(0) = 0$. BÚNO můžeme zavést operátor $\hat A(t) = - \hat B(t) / t$ pro $t \neq 0$ a libovolný pro $t=0$. Dosazením tohoto operátoru do předchozí rovnice dostáváme požadované
    \begin{equation*}
        \hat U(t) = \e{-\i \, \hat A(t) \, t}
    \end{equation*}
\end{proof}

\begin{remark}
Proč nás tolik zajímá zrovna tvar $\e{-\i \, \hat A(t) \, t}$ zjistíme v zápětí. Obzvlášť v případě $\hat A(t) = \konst$ má totiž operátor $\hat A$ důležitý fyzikální význam.
\end{remark}

\begin{theorem}[TDSE]
    Nechť je dynamika systému nezávislá na čase, potom existuje samoadjugovaný operátor $\hat H(t)$, pro který platí tzv. časová Schrödingerova rovnice:
    \begin{equation*}
        \i \; \dd{}{t} \! \ket{\psi(t)} = \hat H(t) \ket{\psi(t)}
    \end{equation*}
\end{theorem}
\begin{proof}
    Z nezávislosti dynamiky systému na čase víme, že platí
    \begin{equation*}
        \ket{\psi(t)} = \e{-\i \, \hat A(t) \, t} \ket{\psi}.
    \end{equation*}
    Když rovnici zderivujeme podle času, dostaneme
    \begin{equation*}
        \dd{}{t} \! \ket{\psi(t)} = -\i \; \left( \hat A(t) + t \, \bigdot{\hat A}(t) \right) \; \underbrace{ \e{-\i \, \hat A(t) \, t} \ket{\psi} \vph{\big|}}_{\ket{\psi(t)}}
    \end{equation*}
    Zavedeme operátor $\hat H(t)$ předpisem
    \begin{equation*}
        \hat H(t) = \hat A(t) + t \, \bigdot{\hat A}(t)
    \end{equation*}
    a ukážeme, že je samoadjugovaný.
    \begin{equation*}
        \hat H^+(t)
        = \big( \hat A(t) + t \, \bigdot{\hat A}(t) \big)^+
        = \hat A^+(t) + t \, \big(\bigdot{\hat A}(t) \big)^+
        = \hat A^+(t) + t \, \big(\hat A^+(t) \big)^{\bigdot{}}
        = \hat A(t) + t \, \bigdot{\hat A}(t)
        = \hat H(t)
    \end{equation*}
\end{proof}

\begin{definition}[Hamiltonián]
    Operátor $\hat H(t)$ z předchozí věty nazýváme hamiltonián.
\end{definition}

\begin{corollary}[Evoluce skleronomního systému]
    \label{evoluce-skleronomniho-systemu}
    Nechť je dynamika systému nezávislá na čase a hamiltonián $\hat H(t)$ je konstantní, potom
    \begin{equation*}
        \ket{\psi(t)} = \e{-\i \, \hat{H} \, t} \ket{\psi}.
    \end{equation*}
\end{corollary}
\begin{proof}
    Z definice hamiltoniánu máme
    \begin{equation*}
        \hat H = \hat A(t) + t \, \bigdot{\hat A}(t) = \konst
    \end{equation*}
    Rozmyslete si, že zvolíme-li nějakou bázi a pomocí ní rovnici vyjádříme maticově, každý element bude nějaká funkce času, pro kterou platí
    \begin{equation*}
        f(t) + t \, f'(t) = \konst
        \iff
        t \, f''(t) + 2 f'(t) = 0
        \iff
        f(t) = \frac{C_1}{t} + C_0.
    \end{equation*}
    I samotný operátor $\hat A(t)$ tedy musí být ve tvaru
    \begin{equation*}
        \hat A(t) = \frac{1}{t} \hat A_1 + \hat A_0
    \end{equation*}
    Protože ale požadujeme, aby byl časový vývoj systému spojitý a $\hat A(t) \, t$ bylo v čase $t=0$ nulové, musí nutně platit $\hat A_1 = 0$. Platí tedy $\hat A(t) = \hat A_0 = \konst$ a konečně $\hat H = \hat A$.
\end{proof}

\begin{definition}[Stacionární stav]
    Mějme stav $\ket{\psi(t)}$, pro který v každém čase $t$ platí
    \begin{equation*}
        \abs{\braket{\psi}{\psi(t)}}^2 = 1.
    \end{equation*}
    Takový stav nazýváme stacionárním stavem systému.
\end{definition}

\begin{lemma}
    Mějme stacionární stav $\ket{\psi(t)}$, potom existuje taková funkce $E(t) \in \R$, že
    \begin{equation*}
        \ket{\psi(t)} = \e{-\i \, E(t) \, t} \ket{\psi}.
    \end{equation*}
\end{lemma}
\begin{proof}
    Z definice stacionárního stavu máme
    \begin{equation*}
        \abs{\braket{\psi}{\psi}}^2
        = 1
        = \abs{\braket{\psi}{\psi(t)}}^2
        = \abs{\bra{\psi} \hat{U}(t) \ket{\psi}}^2.
    \end{equation*}
    Opět z rovnosti absolutních hodnot vyplývá, že existuje funkce $s(t) \in \R$ taková, že
    \begin{align*}
        \braket{\psi}{\psi}
        &= \bra{\psi} \hat{U}(t) \ket{\psi} \e{\i s(t)},
        \\
        \bra{\psi} \e{-\i s(t)} \ket{\psi}
        &= \bra{\psi} \hat{U}(t) \ket{\psi}.
    \end{align*}
    Porovnáním stran dostaneme
    \begin{equation*}
        \hat U(t) = \e{-\i s(t)} \, \1.
    \end{equation*}
    Protože navíc musí platit, že $\hat U(0) = \1$, můžeme si již tradičně zavést funkci $E(t) = s(t) / t$. Po dosazení získáme
    \begin{equation*}
        \hat{U}(t) = \e{-\i \, E(t) \, t} \, \1.
    \end{equation*}
\end{proof}

\begin{theorem}[TISE]
    Nechť jsou dynamika systému a hamiltonián $\hat H$ nezávislé na čase. Potom jsou následující výroky ekvivalentní:
    \begin{enumerate}
        \item Stav $\ket{\psi(t)}$ je stacionární
        \item Platí tzv. bezčasová Schrödingerova rovnice
    \end{enumerate}
    \begin{equation*}
        \hat{H} \ket{\psi} = E \ket{\psi}.
    \end{equation*}
\end{theorem}
\begin{proof}
    Nejprve ukážeme $1 \Rightarrow 2$. Ze stacionarity stavu $\ket{\psi(t)}$ vyplývá
    \begin{equation*}
        \ket{\psi(t)} = \e{-\i \, E(t) \, t} \ket{\psi}.
    \end{equation*}
    Naopak z $\hat H(t) = \konst$ plyne (viz \ref{evoluce-skleronomniho-systemu})
    \begin{equation*}
        \ket{\psi(t)} = \e{-\i \, \hat H \, t} \ket{\psi}.
    \end{equation*}
    Porovnáním zjistíme, že $E(t)$ musí být konstantní. Dosazením do časové Schrödingerovy rovnice potom dostaneme
    \begin{align*}
        \i \; \dd{}{t} \! \e{-\i E t} \ket{\psi}
        &= \hat H \; \e{-\i E t} \ket{\psi}
        \\
        \i \; (-\i) \; E \; \e{-\i E t} \ket{\psi}
        &= \e{-\i E t} \; \hat H \ket{\psi}
        \\
        E \ket{\psi}
        &= \hat H \ket{\psi}
    \end{align*}

    Nyní dokážeme implikaci opačným směrem. Opět použijeme výsledek z \ref{evoluce-skleronomniho-systemu}, tedy pokud dynamika systému a hamiltonián nezávisí na čase, potom platí:
    \begin{equation*}
        \ket{\psi(t)} = \e{\i \hat H t} \ket{\psi}
    \end{equation*}
    Rozepíšeme exponenciálu operátoru z definice a použijeme platnost bezčasové Schrödingerovy rovnice
    \begin{equation*}
        \ket{\psi(t)}
        = \e{-\i \hat H t} \ket{\psi}
        = \sum_{n = 0}^\infty \frac{1}{n!} (-\i \hat H t)^n \ket{\psi}
        = \sum_{n = 0}^\infty \frac{1}{n!} (-\i t)^n \hat H^n \ket{\psi}
        = \sum_{n = 0}^\infty \frac{1}{n!}(-\i t)^n E^n \ket{\psi}
        = \e{-\i E t} \ket{\psi}
    \end{equation*}
    Vidíme, že skutečně
    \begin{equation*}
        \abs{\braket{\psi}{\psi(t)}}^2
        = \abs{\bra{\psi} \; \e{-\i E t} \ket{\psi}}^2
        = \abs{\braket{\psi}{\psi}}^2
        = 1.
    \end{equation*}
\end{proof}

\subsection{Příklady použití}
TODO.

\section{Feynmanův integrál}
TODO.
\begin{itemize}
    \item Fyzikální motivace
    \item Wienerův proces
    \item Wienerova míra a integrál
    \item Feynmanův integrál
\end{itemize}

\end{document}
